\documentclass{article}
\newcommand\lmsuite{{\em lmsuite }}
\title{\lmsuite User's Manual\\
  for version $0.92$}
\author{}
\begin{document}
\maketitle
\tableofcontents
\section{Introduction}
LATTE/MUSE Numerical Suite ({\em lmsuite}) is a Traveling Wave Tube (TWT)
code developed as a part of my Ph.D. thesis research
\cite{wohlbier:phdthesis}. The code can solve three different
1-d nonlinear multifrequency TWT models: LATTE, MUSE, and S-MUSE.
LATTE (Lagrangian TWT Equations)~\cite{wohlbier:muse02}
is a ``large-signal'' code of the
variety originally developed for single frequency
by Nordsieck~\cite{nordsieck:lsbtwa53} and later extended
to multifrequency by El-Shandwily~\cite{el-shandwily:amstwa65}
and Giarola~\cite{giarola:msotwt68}. The MUSE (multifrequency
spectral Eulerian) and S-MUSE (simplified-MUSE) models~\cite{wohlbier:muse02}
are nonlinear models derived for their unique spectral characteristics.
Generally MUSE and S-MUSE are not accurate in saturation due to their
inability to predict electron overtaking effects, however due to their
spectral structure they are very useful for gaining insights into nonlinear
TWT physics~\cite{wohlbier:muse02, wohlbier:growth02, wohlbier:ampm03, wohlbier:hi03, wohlbier:phdthesis}.
Together the three models comprise a ``suite'' hopefully useful for
TWT education, research, and design.

\section{Acknowledgements}
Thanks to Mark Converse for comments on structuring the input deck
and being the first beta tester. Thanks to
Michel Olagnon for helping restructure the LATTE portion of the
code to run faster.

Thanks to my research advisors Professor John Booske
and Professor Ian Dobson for their advice and encouragement. Thanks to
the Innovative Microwave Vacuum Electronics MURI and AFOSR for funding
support.

\section{Physics}
THIS SECTION IS IN PROGRESS. For general model derivations see
\cite{wohlbier:muse02}. Details of code implementations including losses,
normalizations, and slowly varying envelope approximations are in
\cite{wohlbier:phdthesis} and will appear here soon.
\begin{itemize}
\item geometry
\item models
\end{itemize}
\subsection{Sheath helix model}
The sheath helix model currently implemented is the simplest one
available \cite[pg.~??39--41??]{rowe:newip65}. It does not account for
a conducting barrel and should only be used for qualitative
purposes. The equations computed are the cold circuit phase velocity
\begin{eqnarray}\label{eq:phase_velocity}
  v_{\rm ph} & = & \frac{c}{\cot \psi} \left[ \frac{I_0(\gamma r_{\rm h})
      K_0(\gamma r_{\rm h})}{I_1(\gamma r_{\rm h}) K_1(\gamma r_{\rm
      h})}\right]^{1/2},
\end{eqnarray}
and the cold circuit interaction impedance
\begin{eqnarray}\label{eq:impedance}
  K & = & \sqrt\frac{\mu_0}{\epsilon_0} \frac{\cot \psi}{2 \pi}
  \left[I_0(\gamma r_{\rm h}) K_0(\gamma r_{\rm h}) I_1(\gamma r_{\rm
  h}) K_1(\gamma r_{\rm h})\right]^{1/2}.
\end{eqnarray}
In (\ref{eq:phase_velocity}) and (\ref{eq:impedance}) we have
\begin{eqnarray}
  \label{eq:gamma}
  \gamma^2 & = & k^2 \cot^2 \psi \frac{I_1(\gamma r_{\rm h})
    K_1(\gamma r_{\rm h})}{I_0(\gamma r_{\rm h}) K_0(\gamma r_{\rm h})}\\
  k^2 & = & \omega^2 \mu_0 \epsilon_0\\
  \cot \psi & = \frac{2 \pi r_{\rm h}}{z_{\rm h}},
\end{eqnarray}
where $r_{\rm h}$ is the helix radius, $z_{\rm h}$ is the turn to turn
spacing (entered into the code as ``helix pitch''), and
$\mu_0$, $\epsilon_0$ and $c$ are the permeability,
permittivity, and the speed of light of free space respectively.
\subsection{Space charge reduction factor}
The formula for computing space charge reduction factor is taken
from \cite{antonsen:twtdnde98}. Using the Bessel function identity
\begin{eqnarray}
  \frac{d}{dz}K_0(z) & = & - K_1(z)
\end{eqnarray}
we have for the factor $R_{\rm sc}$
\begin{eqnarray}
  \lefteqn{R_{\rm sc} =} & & \nonumber\\
  && 1 + \frac{2}{r_{\rm bo}^2 - r_{\rm bi}^2}
  \left\{\left[\rule{0mm}{5mm}r_{\rm bo}I_1(\kappa r_{\rm bo})
      - r_{\rm bi}I_1(\kappa r_{\rm bi})\right]
    \left[\frac{K_0(\kappa r_{\rm h})}{I_0(\kappa r_{\rm h})}
      \left(\rule{0mm}{5mm}r_{\rm bi}I_1(\kappa r_{\rm bi})
        - r_{\rm bo}I_1(\kappa r_{\rm bo})
      \right)\right.\right.\nonumber\\
  && \left.\left. - r_{\rm bo} K_1(\kappa r_{\rm bo})\rule{0mm}{5mm}\right]
    - r_{\rm bi} I_1(\kappa r_{\rm bi})
    \left[\rule{0mm}{5mm}r_{\rm bi}K_1(\kappa r_{\rm bi})
      - r_{\rm bo}K_1(\kappa r_{\rm bo})\right]\right\}
\end{eqnarray}
where $r_{\rm bo}$ is the outer beam radius, $r_{\rm bi}$ is the inner
beam radius, $r_{\rm h}$ is the helix radius, and
\begin{eqnarray}
  \kappa^2 & = & \beta_{e,c}^2 - \frac{\omega^2}{c^2}.
\end{eqnarray}
For the term $\beta_{e,c}$ one can use either the electron stream 
wavenumber $\beta_e$
\begin{eqnarray}
  \beta_e & = & \frac{\omega}{u_0}
\end{eqnarray}
where $u_0$ is the dc beam velocity, or the circuit wavenumber $\beta_c$
\begin{eqnarray}
  \beta_c & = & \frac{\omega}{\tilde v_{{\rm ph}\ell}}
\end{eqnarray}
where $\tilde v_{{\rm ph}\ell}$ is the frequency dependent cold circuit
phase velocity. To use $\beta_c$ one sets {\tt use\_beta\_c = true}
in the {\tt \&dispersion} namelist. However, when {\tt use\_beta\_c = true},
$\beta_e$ is still used for frequencies that do not have a phase velocity
specified.

\section{Running \lmsuite}
To download executables and source code for \lmsuite go to\\
{\tt http://www.lmsuite.org} and follow
the instructions provided.

To run {\em lmsuite}:
\begin{enumerate}
\item open a command shell (Windows, Linux, or Unix)
\item change to the {\tt LMSuitev0.92} directory
\item type {\tt lmsuite} at the shell prompt.
\end{enumerate}
The program
reads files in the directory {\tt ./LMSuitev0.92/inputs/} to determine
what is to be computed. Possible files in {\tt ./inputs/} are
{\tt lmsuite.nml} (main
input deck),
{\tt scan\_data.nml} (optional input deck containing scan
information),
{\tt movies.nml} (optional input deck containing information to
generate ``frame'' files used for animation),
{\tt circuit.in} (optional file defining TWT circuit),
{\tt frequencies.in} (optional file specifying input frequencies, input powers,
and input phases), {\tt dispersion.in} (optional file specifying TWT
dispersion properties), and {\tt losses.in} (optional file specifying
circuit losses). For details on using these files see
Section~\ref{sec:namelists}. The output data is written to the directory
{\tt ./LMSuitev0.92/outputs/}. See Section \ref{ss:sample_namelists}
for a description of the sample namelists included with the distribution.

Generally one can run the code in either ``single pass mode'' or ``scan mode''
as described below.

\subsection{Single pass mode}\label{ss:spm}
The single pass mode solves the TWT equations for a set of
fixed input parameters.
Single pass mode is chosen by setting {\tt num\_scan\_namelists}~$= 0$ in the
{\tt \&run} namelist.

Available plotting routines in single pass mode are:
\begin{itemize}
\item {\tt power\_out\_vs\_freq}: Circuit power at TWT output versus frequency
  for all circuit frequencies. Output data file: {\tt pow\_vs\_freq.X.sp.dat}.
\item {\tt phase\_vs\_freq}: Voltage phase at TWT output versus frequency
  for all circuit frequencies. Output data file:
  {\tt phase\_vs\_freq.X.sp.dat}.
\item {\tt circuit\_power\_vs\_z}: Circuit power versus axial distance $z$ for
  all circuit frequencies. Output data file: {\tt pow\_vs\_z.X.sp.dat}.
\item {\tt magnitude\_vs\_z}: Magnitude of circuit voltage, circuit current,
  space charge field, beam velocity, or beam density versus axial
  distance $z$ for all circuit or space charge frequencies (depending on
  whether the variable is a circuit variable or a beam variable). Variable
  chosen by integer $1-6$ respectively. See {\tt \&output\_data} namelist
  for details. Output data file: {\tt mag\_vs\_z.X.sp.dat}.
\item {\tt phase\_vs\_z}: Phase of circuit voltage, circuit current, space
  charge field, beam velocity, or beam density versus axial distance $z$ for
  all circuit or space charge frequencies (depending on whether the variable
  is a circuit variable or a beam variable). Variable chosen by
  integer $1-6$ respectively. See {\tt \&output\_data} namelist for details.
  Output data file: {\tt phase\_vs\_z.X.sp.dat}.
\item {\tt phase\_differences\_vs\_z}: Phase difference between two of:
  circuit voltage, circuit current, space charge field, beam velocity,
  or beam density versus axial distance $z$ for all circuit or space charge
  frequencies (depending on whether the variables are circuit
  variables or a beam variables). Variables chosen by integers $1-6$.
  See {\tt \&output\_data} namelist for details.
  Output data file: {\tt phase\_diff\_vs\_z.X.sp.dat}.
\item {\tt disk\_orbits\_vs\_z}: not implemented
\item {\tt conserved\_quantity\_vs\_z}: not implemented
\item {\tt beam\_energy\_vs\_z}: not implemented
\item {\tt dc\_beam\_vel\_vs\_z}: The average electron beam velocity as
  as function of axial distance $z$.
  Output data file: {\tt dcbeam\_vs\_z.X.sp.dat}.
\item {\tt phase\_space} (LATTE only): Create particle phase space plots
  at a specified number of times.
  Output data file: {\tt phase\_space.M.dat} where M corresponds to the
  location of the {\tt time\_array}.
\end{itemize}
In the above file names {\tt X} is {\tt L, M,} or {\tt S} depending
on whether LATTE, MUSE or S-MUSE generates the data. The tag {\tt .sp.}
in the file name represents ``single pass.''

\subsection{Scan mode}
In contrast to the single pass mode one can have a simulation scan various
input parameters, solving the TWT equations for each set of input
parameters.

\subsubsection{Single parameter scans}
For a single parameter scan a single input parameter is scanned over
a specified range. Each single parameter scan adds $1$ to the value of
{\tt num\_scan\_namelists} in the
{\tt \&run} namelist. The scan has its own {\tt \&scan} namelist
in the file {\tt scan\_data.nml} and the type is specified by the
{\tt scanID}. Full details for specifying
scans are given in Section~\ref{ss:scan_data}. The types of single parameter
scans and the output each can provide are:
\begin{enumerate}
\item Input power scan ({\tt scanID}~$=1$): For a chosen frequency the
  input power is varied over a range.
  This scan is used for making the widely used AM/AM and AM/PM curves.\\
  Available plotting routines in the input power scan are:
  \begin{itemize}
  \item {\tt power\_out\_vs\_power\_in}: Output power versus input power at
    either the frequency specified in the scan or all frequencies.
    Output data file: {\tt pout\_vs\_pin.X.N.dat}.
  \item {\tt gain\_vs\_power\_in}: Gain versus input power at either
    the frequency specified in the scan or all frequencies.
    Output data file: {\tt gain\_vs\_pin.X.N.dat}.
  \item {\tt phase\_out\_vs\_power\_in}: Output voltage phase versus input
    power at either the frequency specified in the scan or all frequencies.
    Output data file: {\tt phase\_vs\_pin.X.N.dat}.
  \item All of the {\tt *\_vs\_z} plots (see Section \ref{ss:spm} for a
    listing). For each of the input power values, a full set of
    {\tt *\_vs\_z} curves are given.
  \end{itemize}
\item Input phase scan ({\tt scanID}~$=2$): For a chosen frequency the
  input phase is varied over a range. This scan is useful for studying
  harmonic/intermodulation injection where getting the critical input phase
  is neccessary to obtain reduction of the harmonic/intermodulation outputs.\\
  Available plotting routines in the input power scan are:
  \begin{itemize}
  \item {\tt power\_out\_vs\_phase\_in}: Output power versus input phase at
    the frequency specified in the scan.
    Output data file: {\tt pout\_vs\_phasein.X.N.dat}.
  \item All of the {\tt *\_vs\_z} plots (see Section \ref{ss:spm} for a
    listing). For each of the input phase values, a full set of
    {\tt *\_vs\_z} curves are given.
  \end{itemize}
\item Frequency scan ({\tt scanID}~$=3$): For a chosen input
  {\tt frequency\_integer} {\em array location}
  (namelist {\tt \&frequency\_list}), scan the frequency over a range of
  values. That is, scan the value of frequency for a particular array
  element in the frequency array {\tt frequency\_integer}.\\
  Available plotting routines in the frequency scan are:
  \begin{itemize}
  \item {\tt power\_out\_vs\_freq}: Output power at the {\em scanned}
    frequency versus the scanned frequency OR output power at a {\em different
    frequency} versus the scanned frequency.
    Output data file: {\tt pout\_vs\_freq.X.N.dat}.
  \item {\tt gain\_vs\_freq}: Gain at the {\em scanned}
    frequency versus the scanned frequency.
    Output data file: {\tt gain\_vs\_freq.X.N.dat}.
  \item {\tt phase\_vs\_freq}: Output voltage phase at the {\em scanned}
    frequency versus the scanned frequency OR output voltage phase
    at a {\em different frequency} versus the scanned frequency.
    Output data file: {\tt pout\_vs\_freq.X.N.dat}.
  \end{itemize}
\item Dispersion parameter scan ({\tt scanID}~$=4$): At a particular frequency
  scan phase velocity, interaction impedance, loss, or space charge reduction
  factor.\\
  Available plotting routines in the dispersion parameter scan are:
  \begin{itemize}
  \item All of the {\tt *\_vs\_z} plots (see Section \ref{ss:spm} for a
    listing). For each of the dispersion parameter values, a full set of
    {\tt *\_vs\_z} curves are given.
  \end{itemize}
\item Beam parameter scan: not implemented
\end{enumerate}
In the above file names {\tt X} is {\tt L, M,} or {\tt S} depending
on whether LATTE, MUSE or S-MUSE generates the data, and {\tt N} is
an integer corresponding to the scan number, i.e., the order that
the scan appears in the file {\tt scan\_data.nml}.

In principle one can run as many scans as desired. Presently the maximum
number is $10$ (set by {\tt max\_scans} in the source file
{\tt parameters.f90}), but this can easily be changed for future versions.


\subsubsection{Two parameter scans}
A two parameter scan connects two {\tt \&scan} namelists so that
two different parameters are scanned simultaneously.
Two {\tt \&scan} namelists must be provided for each two parameter
scan desired. The first of
these two namelists must have {\tt two\_parameter = true}, where the second
namelist must have {\tt two\_parameter = false}. Correspondingly
a count of $2$ must be added to {\tt num\_scan\_namelists} in the
{\tt\&run} namelist in {\tt lmsuite.nml}.

As a general rule two parameter scans provide data that will generate
parameterized curves. The first scan (i.e.~the scan with
{\tt two\_parameter = true}) should be over the variable that you wish
to have as your independent coordinate axis in graphs, and the second scan
should be
over the variable that will parameterize the curves.

Not all scan types (i.e.~{\tt scanID}s) may be used together.
The allowable combinations of {\tt scanID}s are given below.
Note that the ordering of the scans in the list below must be the same
ordering entered into the namelist.
\begin{itemize}
\item Input power scan ({\tt scanID = 1}) and dispersion parameter scan
  ({\tt scanID = 4}).
  Available plotting routines for this two parameter scan are\\
  {\tt power\_out\_vs\_power\_in}, {\tt gain\_vs\_power\_in}, and
  {\tt phase\_out\_vs\_power\_in}.
\end{itemize}


\subsection{Namelists}\label{sec:namelists}
\subsubsection{\tt lmsuite.nml}
NAMELIST DESCRIPTIONS ARE CURRENTLY GIVEN ONLY IN THE PROVIDED NAMELIST
FILES. AS SOON AS THE NAMELISTS SEEM TO HAVE MATURED, THE INFORMATION WILL
BE TRANSFERRED TO THIS SECTION.

\subsubsection{\tt scan\_data.nml}\label{ss:scan_data}
NAMELIST DESCRIPTIONS ARE CURRENTLY GIVEN ONLY IN THE PROVIDED NAMELIST
FILES. AS SOON AS THE NAMELISTS SEEM TO HAVE MATURED, THE INFORMATION WILL
BE TRANSFERRED TO THIS SECTION.

\subsubsection{\tt movies.nml}\label{ss:movies}
NAMELIST DESCRIPTIONS ARE CURRENTLY GIVEN ONLY IN THE PROVIDED NAMELIST
FILES. AS SOON AS THE NAMELISTS SEEM TO HAVE MATURED, THE INFORMATION WILL
BE TRANSFERRED TO THIS SECTION.

\subsubsection{The sample namelists}\label{ss:sample_namelists}
The namelists ({\tt *.nml}) and {\tt *.in} files provided with the code
are based on the TWT circuit found in \cite{abe:clhtwtc00}.
The default run is a single pass of the slowly varying envelope version
of LATTE for $1.538\,$GHz. A scan namelist is also provided to create
a figure similar to Fig.~3 of \cite{abe:clhtwtc00}.
The circuit and beam parameters are only approximate, so we do
not expect to be able to reproduce exactly the results of~\cite{abe:clhtwtc00}.
Comments are provided below on certain aspects of the input files to
help the new user get started.

\begin{itemize}
\item {\tt \&user\_interface}
  \begin{itemize}
  \item This namelist is presently set to maximum verboseness. Setting all of
    these flags to false forces the code to run in ``silent mode.''
  \item Once one is comfortable that they are entering the proper data,
    they can set {\tt echo\_namelists = false}. However, when preparing
    a new namelist this flag can be {\em very} useful in determining whether
    you are entering the data that you intend to. Setting
    {\tt echo\_namelists = true} should always be your
    first recourse if the code is not doing what you think it should do.
  \end{itemize}
\item {\tt \&run}
  \begin{itemize}
  \item Since the TWT circuit has a substantial sever (see
    Fig.~1 of \cite{abe:clhtwtc00}) one has to use the slowly varying envelope
    version of the models ({\tt svea = true}) due to reflection issues.
    Since only LATTE presently
    has the slowly varying envelope formulation available, we choose only to
    run LATTE ({\tt select\_code = 'L'}).
  \item For single pass mode one has {\tt num\_scan\_namelists = 0}. To run
    the input power scan change {\tt num\_scan\_namelists = 1}. The input
    power scan produces data that when graphed resemble Fig.~3(a)
    of \cite{abe:clhtwtc00}.
    Changing {\tt read\_from\_file = true} in {\tt \&frequency\_list} and
    changing
    {\tt int\_param\_1 = 1750} in the {\tt \&scan} namelist should
    approximate Fig.~3(b) of \cite{abe:clhtwtc00}.
  \end{itemize}
\item {\tt \&output\_data}
  \begin{itemize}
  \item Since {\tt plot\_dispersion = true} files will be generated in
    {\tt ./outputs/} containing the circuit information. Graph the contents
    of these files to see
    how we have approximated the circuit of Fig.~1 in \cite{abe:clhtwtc00}.
    {\tt vph.dat} contains phase velocity data,
    {\tt impedance.dat} has interaction impedance data,
    {\tt loss.dat} contains loss information, and
    {\tt scrf.dat} contains space charge reduction factor data.
  \end{itemize}
\item {\tt \&frequency\_list}
  \begin{itemize}
  \item To be able to resolve frequencies down to $1.0\,$MHz we set\\
    {\tt base\_frequency}$= 1.0{\rm e}6\,$Hz. Therefore for $1.538\,$GHz the
    frequency integer is $1538$. Changing {\tt read\_from\_file = true}
    changes the frequency to $1.75\,$GHz since this is the frequency specified
    in\\ {\tt frequencies.in}.
  \item We set {\tt highest\_order\_IMP = 2} so that the second harmonic is
    computed. One could include the $3^{\rm rd}$ harmonic {\em in the beam}
    by changing {\tt highest\_order\_IMP = 3} and increasing the
    {\tt max\_space\_charge\_freq} to $3900$. However, unless
    additional dispersion data is supplied
    the {\tt max\_ckt\_freq} must not be increased.
  \end{itemize}
\item {\tt \&circuit}
  \begin{itemize}
  \item Since {\tt read\_from\_file = true} circuit data is read from
    {\tt circuit.in}.
  \end{itemize}
\item {\tt \&dispersion}
  \begin{itemize}
  \item There are 10 sections in this circuit. There are three ``pull
    turns,'' two uniform sections, a tapered section, another small
    uniform section, and finally three pull turns. To get the tapered
    section set {\tt inrplt\_sections = true} and list section number
    6 in {\tt interpol\_sects\_list}. Sections 6 and 7 have identical
    parameters so that paramters are continuous across their
    interface. This is an example of a moderately complicated
    circuit.
  \item Since {\tt use\_antonsen\_formula = true} one can put $0.0$'s in
    for all {\tt space\_charge\_redux} values.
  \end{itemize}
\item {\tt \&losses}
  \begin{itemize}
  \item Frequency independent loss is entered by setting two loss frequencies,
    one below {\tt min\_ckt\_freq} and one above {\tt max\_ckt\_freq}.
  \item To compare how loss data is entered when {\tt read\_from\_file =
      true}, see {\tt losses.in}.
  \end{itemize}
\item {\tt \&numerical\_data}
  \begin{itemize}
  \item The numbers in this namelist are good starting points for most
    simulations.
  \end{itemize}
\end{itemize}

\subsection{Problems running?}
THIS SECTION IS IN PROGRESS. PLEASE REPORT SUGGESTIONS.
\begin{enumerate}
\item Q: The program crashes while reading data. A:
\item Q: The program crashes while reading data. Error: ``Too many values
  are specified for the item name X in the NAMELIST input data.''\\
  A: Code built by the Lahey compiler does not like blank lines in namelists.
  Remove any blank lines or lines WITH ONLY COMMENTS from your namelists.
\item Q: Something is fishy. A: Try echoing input to make sure you are putting
  in what you think you are putting in.
\end{enumerate}





\section{Compiling}
See {\tt http://www.lmsuite.org}. Instructions
will eventually be transferred to here.

\subsection{Features accessible only by recompiling}
This section describes some features in the code that are so specialized
that they do not warrant control by namelist variables. To access the features
one must edit the code and recompile.

\subsubsection{Create only harmonic frequencies}
The algorithm for computing
intermodulation products is {\em very} inefficient if one only wants to
compute harmonics of a fundamental. For modest numbers of harmonics this
is not a problem, however for large numbers of harmonics, e.g. $> 40$,
the time it takes to compute the harmonics becomes on the order of several
minutes. When the number of harmonics is even larger, e.g. $>100$ then
it takes unacceptably long to compute the harmonics. The code contains
a function that computes the harmonics using the simple harmonic formula,
rather than the more general intermodulation frequency computation,
which makes simulations using hundreds of space charge harmonics possible.

To use this function open the file {\tt initialize.f90}, comment
out the line {\tt call create\_frequency\_array(.true.)}, and comment
in the line\\
{\tt call create\_harmonic\_array(.true.)}. Recompile as
described earlier in this section.


\section{Contributing}
See {\tt http://www.lmsuite.org}. Instructions
will eventually be transferred to here.


\section{To Do List}
\subsection{Testing}
Below are input scenarios for which the code has not been extensively tested.
The code was written to handle the cases, but without testing one cannot
be sure that the cases are handled correctly. If you are getting unexpected
results from the code, have a look at this list to see if you are in
``uncharted waters.'' Alternatively, if you would like to help test the
code, consider putting it through its paces with any of the following
input scenarios.
\begin{enumerate}
\item Space charge frequency range wider than circuit frequency range. Some
  aspects to look at include plotting magnitudes and phases of the various
  quantities. For example if choose to plot magnitude or phase of a circuit
  quantity are only circuit frequencies included in the output, if choose to
  plot a beam quantity do all space charge frequencies appear in the output.
\item The space charge reduction formula for an annular beam has not been
  tested.
\item The dispersion scan ({\tt scanID = 4}) has not been tested extensively,
  particularly for loss and space charge reduction factor.
\end{enumerate}
\subsection{Models}
\begin{enumerate}
\item LATTE
  \begin{enumerate}
  \item scan routines: beam parameter scan
  \item plot routines:
    phase\_vs\_z (velocity only), phase\_differences\_vs\_z (velocity only),
    disk\_orbits\_vs\_z, conserved\_quantity\_vs\_z, beam\_energy\_vs\_z
  \end{enumerate}
\item MUSE
  \begin{enumerate}
  \item check that ${\bf V}$ matrix can be larger than number of space charge
    frequencies
  \item slowly varying envelope implementation
  \item scan routines: beam parameter scan
  \item plot routines:
    conserved\_quantity\_vs\_z, beam\_energy\_vs\_z
  \end{enumerate}
\item S-MUSE
  \begin{enumerate}
  \item slowly varying envelope implementation
  \item scan routines: beam parameter scan
  \item plot routines:
    conserved\_quantity\_vs\_z, beam\_energy\_vs\_z
  \end{enumerate}
\end{enumerate}
\subsection{Physics}
\begin{enumerate}
\item modulated electron beam (need initial values of space charge electric
  field $E$?)
\item 2-d beam stuff, thermal spread
\end{enumerate}
\subsection{Helix}
\begin{enumerate}
\item Sheath helix solver
\item Tape helix solver
\item Loss model?
\end{enumerate}
\subsection{Beam}
\begin{enumerate}
\item Antonsen correction to space charge reduction factor
\item DC space charge depression
\item Relativistic effects
\end{enumerate}
\subsection{Namelist}
\begin{enumerate}
\item Test code defaults, i.e. remove {\tt lmsuite.nml}
\item Put in dispersion and loss frequencies in real numbers (?) (Hz?)
\end{enumerate}
\subsection{Plotting}
\begin{enumerate}
\item Roll all single pass {\tt *\_vs\_freq} plots into one routine.
\end{enumerate}
\subsection{Numerical}
\begin{enumerate}
\item reduce arrays for circuit and space charge field to their minimum
  number of elements
\item implement adaptive stepsize (dormand-prince)
\item implement linear multistep method (adams-bashford)
\end{enumerate}
\subsection{Compiler}
\begin{enumerate}
\item study optimization flags for all supported platforms
\item g95 build
\end{enumerate}

\subsection{Documentation}
\begin{enumerate}
\item Register copyright, see http://www.gnu.org/licenses/gpl-howto.html
  for instructions.
\item compilation and installation instructions
\item Manual
\end{enumerate}

\bibliographystyle{plain}
\bibliography{../../../ResearchDocs/Refs.BibTeX/strings.bib,../../../ResearchDocs/Refs.BibTeX/others.bib,../../../ResearchDocs/Refs.BibTeX/papers.bib,../../../ResearchDocs/Refs.BibTeX/books.bib}

\end{document}
